\documentclass{article}

\usepackage[utf8]{inputenc}
\usepackage{comment}
\usepackage{todonotes}
\usepackage{msc5}
\usepackage{listings}
\usepackage{url}
\usepackage{tikz-qtree}
\usepackage{graphicx}
\usepackage{paralist}
\usepackage{algorithm,algorithmic}
\usepackage{amsmath, amssymb, amsthm, amsfonts, caption, stmaryrd, mathtools, syntax, mdframed}
\usepackage{cryptocode}
\usepackage{hyperref}
\hypersetup{
    colorlinks,
    citecolor=black,
    filecolor=black,
    linkcolor=black,
    urlcolor=black
}
\title{QUIC Handshake Key Discard Alternatives}
\author{Eric Rescorla}
\date{\today}

\begin{document}

\begin{figure}[th!]
\begin{center}
\resizebox{0.8\columnwidth}{!}{\begin{msc}{}
  \drawframe{none}
  \setlength{\topheaddist}{0cm}
  \setlength{\instdist}{8.5cm}
  \setlength{\instwidth}{1.75cm}
  \declinst{cl}{}{Client $C$}
  \declinst{sr}{}{Server $S$}

  \mess{\emph{Initial [C]}}{cl}{sr}
  \nextlevel[2]
  \mess{\emph{Initial [SH], Handshsake[EE...SFIN]}}{sr}{cl}
  \nextlevel
  \action*{Server knows 1-RTT keys}{sr}
  \nextlevel[2]
  \mess{\emph{Handshake [CFIN]}}{cl}{sr}
  \nextlevel
  \action*{Client knows 1-RTT keys; Handshake Complete}{cl}
  \action*{Handshake Complete}{sr}
  \nextlevel[2]
  \mess{\emph{1-RTT [DATA]}}{cl}{sr}
  \nextlevel[2]
  \mess{\emph{1-RTT [DATA, ACK]}}{sr}{cl}
  \nextlevel
  \action*{Handshake Confirmed; Drop Handshake Keys}{cl}
  \nextlevel[3]  
  \mess{\emph{1-RTT [ACK]}}{cl}{sr}
  \nextlevel    
  \action*{Handshake Confirmed; Drop Handshake Keys}{sr}
  \nextlevel[2]
  
  \nextlevel[2]
\end{msc}
}
\end{center}
\caption{Basic QUIC handshake (without 0.5 RTT data).}
\label{fig:basic-handshake}
\end{figure}

Figure~\ref{fig:basic-handshake} shows the basic QUIC handshake along
with the various checkpoints. This just for reference and hopefully
non-controversial.




\end{document}

