\documentclass{article}

\usepackage[utf8]{inputenc}
\usepackage{comment}
\usepackage{todonotes}
\usepackage{msc5}
\usepackage{listings}
\usepackage{url}
\usepackage{tikz-qtree}
\usepackage{graphicx}
\usepackage{paralist}
\usepackage{algorithm,algorithmic}
\usepackage{amsmath, amssymb, amsthm, amsfonts, caption, stmaryrd, mathtools, syntax, mdframed}
\usepackage{cryptocode}
\usepackage{hyperref}
\hypersetup{
    colorlinks,
    citecolor=black,
    filecolor=black,
    linkcolor=black,
    urlcolor=black
}
\title{QUIC Handshake Key Discard Alternatives}
\author{Eric Rescorla}
\date{\today}

\begin{document}

\section{Status Quo}
\label{sec:status-quo}


\begin{figure}[h!]
\begin{center}
\resizebox{0.8\columnwidth}{!}{\begin{msc}{}
  \drawframe{none}
  \setlength{\topheaddist}{0cm}
  \setlength{\instdist}{8.5cm}
  \setlength{\instwidth}{1.75cm}
  \declinst{cl}{}{Client $C$}
  \declinst{sr}{}{Server $S$}

  \mess{\emph{Initial [C]}}{cl}{sr}
  \nextlevel[2]
  \mess{\emph{Initial [SH], Handshake[EE...SFIN]}}{sr}{cl}
  \nextlevel
  \action*{Server knows 1-RTT keys}{sr}
  \nextlevel[2]
  \mess{\emph{Handshake [CFIN]}}{cl}{sr}
  \nextlevel
  \action*{Client knows 1-RTT keys; Handshake Complete}{cl}
  \action*{Handshake Complete}{sr}
  \nextlevel[2]
  \mess{\emph{1-RTT [DATA]}}{cl}{sr}
  \nextlevel[2]
  \mess{\emph{1-RTT [DATA, ACK]}}{sr}{cl}
  \nextlevel
  \action*{Handshake Confirmed; Drop Handshake Keys}{cl}
  \nextlevel[3]  
  \mess{\emph{1-RTT [ACK]}}{cl}{sr}
  \nextlevel    
  \action*{Handshake Confirmed; Drop Handshake Keys}{sr}
  \nextlevel[2]
  
  \nextlevel[2]
\end{msc}
}
\end{center}
\caption{Basic QUIC handshake (without 0.5 RTT data).}
\label{fig:basic-handshake}
\end{figure}

Figure~\ref{fig:basic-handshake} shows the basic QUIC handshake along
with the various checkpoints. This just for reference and hopefully
non-controversial.


\section{Never Drop the Keys (PR 3121)}
\label{sec:never-drop-keys}

The basic idea here is to (1) never drop the Handshake keys
and (2) continue to send ACKs for packets you have received
(as agreed upon separately in YUL, you MUST ACK each
packet at least once). The result is that the handshake
goes quiescent even if no application data is ever transmitted,
though you will not reach ``handshake confirmed'' in this
case, as shown in Figure~\ref{fig:never-drop-1}.

\begin{figure}[h!]
\begin{center}
\resizebox{0.8\columnwidth}{!}{\begin{msc}{}
  \drawframe{none}
  \setlength{\topheaddist}{0cm}
  \setlength{\instdist}{8.5cm}
  \setlength{\instwidth}{1.75cm}
  \declinst{cl}{}{Client $C$}
  \declinst{sr}{}{Server $S$}

  \mess{\emph{Initial [C]}}{cl}{sr}
  \nextlevel[2]
  \mess{\emph{Initial [SH], Handshake[EE...SFIN]}}{sr}{cl}
  \nextlevel
  \action*{Server knows 1-RTT keys}{sr}
  \nextlevel[2]
  \mess{\emph{Handshake [CFIN, ACK]}}{cl}{sr}
  \nextlevel
  \action*{Client knows 1-RTT keys; Handshake Complete}{cl}
  \action*{Handshake Complete}{sr}
  \nextlevel[3]
  \mess{\emph{Handshake [ACK]}}{sr}{cl}
  
  \nextlevel[2]
\end{msc}
}
\end{center}
\caption{Never drop keys (no data sent)}
\label{fig:never-drop-1}
\end{figure}


One obvious question is what happens in the case where you
\emph{are} sending data. In the basic case, we would just
expect ACK piggybacking on the data, as shown in Figure~\ref{fig:never-drop-2}

\begin{figure}[h!]
\begin{center}
\resizebox{0.8\columnwidth}{!}{\begin{msc}{}
  \drawframe{none}
  \setlength{\topheaddist}{0cm}
  \setlength{\instdist}{8.5cm}
  \setlength{\instwidth}{1.75cm}
  \declinst{cl}{}{Client $C$}
  \declinst{sr}{}{Server $S$}

  \mess{\emph{Initial [C]}}{cl}{sr}
  \nextlevel[2]
  \mess{\emph{Initial [SH], Handshake[EE...SFIN]}}{sr}{cl}
  \nextlevel
  \action*{Server knows 1-RTT keys}{sr}
  \nextlevel[2]
  \mess{\emph{Handshake [CFIN, ACK]}}{cl}{sr}
  \nextlevel
  \action*{Client knows 1-RTT keys; Handshake Complete}{cl}
  \action*{Handshake Complete}{sr}
  \nextlevel[2]
  \mess{\emph{1-RTT [DATA]}}{cl}{sr}
  \nextlevel[2]
  \mess{\emph{Handshake [ACK], 1-RTT [DATA, ACK]}}{sr}{cl}
  \nextlevel
  \action*{Handshake Confirmed}{cl}
  \nextlevel[3]  
  \mess{\emph{1-RTT [ACK]}}{cl}{sr}
  \nextlevel    
  \action*{Handshake Confirmed}{sr}
  \nextlevel[2]
  
  \nextlevel[2]
\end{msc}
}
\end{center}
\caption{Never drop keys (data sent)}
\label{fig:never-drop-2}
\end{figure}

You can also safely implement implicit ACK. At the point
where you have handshake confirmed, you can stop sending
any of your Handshake DATA frames (though you still
need to ACK any of the peer's Handshake frames, or you
will get deadlocks if the peer doesn't implement implicit
ACKs.)

Kazuho had raised two concerns about this design:

\begin{enumerate}
\item Because handshake confirmed is not guaranteed, key update
  is confusing.
\item It is possible to have outstanding handshake data at the
  point where you want to do migration.
\end{enumerate}

I don't really understand point (1). As noted in
\url{https://github.com/quicwg/base-drafts/issues/3212}, the handshake
confirmed test for key update is redundant, and so we can just re-code
this as ``don't initiate key update for key $n+1$ until you are sure
the other side has key $n$''. And this just lets us have the usual
rule without a special case for the handshake $\rightarrow$ 1-RTT
transition.

Point (2) needs some elaboration. As a general matter, performing
migration while any handshake data is outstanding is a recipe for
bridging the two paths. In the current document this is addressed by
(1) requiring that the client not initiate migration prior to handshake
confirmed and (2) requiring that the 







\end{document}

