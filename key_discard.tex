\documentclass{article}

\usepackage[utf8]{inputenc}
\usepackage{comment}
\usepackage{todonotes}
\usepackage{msc5}
\usepackage{listings}
\usepackage{url}
\usepackage{tikz-qtree}
\usepackage{graphicx}
\usepackage{paralist}
\usepackage{algorithm,algorithmic}
\usepackage{amsmath, amssymb, amsthm, amsfonts, caption, stmaryrd, mathtools, syntax, mdframed}
\usepackage{float}
\usepackage{cryptocode}
\usepackage{hyperref}
\hypersetup{
    colorlinks,
    citecolor=black,
    filecolor=black,
    linkcolor=black,
    urlcolor=black
}
\title{QUIC Handshake Key Discard Alternatives}
\author{Eric Rescorla}
\date{\today}

\begin{document}

\maketitle

\section{Status Quo}
\label{sec:status-quo}


\begin{figure}[H]
\begin{center}
\resizebox{0.8\columnwidth}{!}{\begin{msc}{}
  \drawframe{none}
  \setlength{\topheaddist}{0cm}
  \setlength{\instdist}{8.5cm}
  \setlength{\instwidth}{1.75cm}
  \declinst{cl}{}{Client $C$}
  \declinst{sr}{}{Server $S$}

  \mess{\emph{Initial [C]}}{cl}{sr}
  \nextlevel[2]
  \mess{\emph{Initial [SH], Handshake[EE...SFIN]}}{sr}{cl}
  \nextlevel
  \action*{Server knows 1-RTT keys}{sr}
  \nextlevel[2]
  \mess{\emph{Handshake [CFIN]}}{cl}{sr}
  \nextlevel
  \action*{Client knows 1-RTT keys; Handshake Complete}{cl}
  \action*{Handshake Complete}{sr}
  \nextlevel[2]
  \mess{\emph{1-RTT [DATA]}}{cl}{sr}
  \nextlevel[2]
  \mess{\emph{1-RTT [DATA, ACK]}}{sr}{cl}
  \nextlevel
  \action*{Handshake Confirmed; Drop Handshake Keys}{cl}
  \nextlevel[3]  
  \mess{\emph{1-RTT [ACK]}}{cl}{sr}
  \nextlevel    
  \action*{Handshake Confirmed; Drop Handshake Keys}{sr}
  \nextlevel[2]
  
  \nextlevel[2]
\end{msc}
}
\end{center}
\caption{Basic QUIC handshake (without 0.5 RTT data).}
\label{fig:basic-handshake}
\end{figure}

Figure~\ref{fig:basic-handshake} shows the basic QUIC handshake along
with the various checkpoints. This just for reference and hopefully
non-controversial.

This design is known to have several problems, specifically:

\begin{enumerate}
\item A dropped ACK can cause the client to retransmit CFIN
  indefinitely.
\item It is possible for the server to have handshake data
  outstanding when the client initiates migration. 
\end{enumerate}

We show these scenarios below.

\subsection{Deadlock with Missing ACKs}
\label{sec:deadl-with-miss}

[TODO: Diagram needed]

\subsection{Outstanding Handshake Data During Migration}
\label{sec:outst-handsh-data}

Consider the situation shown in 
\begin{figure}[H]
\begin{center}
\resizebox{0.8\columnwidth}{!}{\begin{msc}{}
  \drawframe{none}
  \setlength{\topheaddist}{0cm}
  \setlength{\instdist}{8.5cm}
  \setlength{\instwidth}{1.75cm}
  \declinst{cl}{}{Client $C$}
  \declinst{sr}{}{Server $S$}

  \mess{\emph{Initial [C]}}{cl}{sr}
  \nextlevel[2]
  \mess{\emph{Initial [SH], Handshake[EE...SFIN]}}{sr}{cl}
  \nextlevel
  \action*{Server knows 1-RTT keys}{sr}
  \nextlevel[2]
  \mess{\emph{Handshake [CFIN]}}{cl}{sr}
  \nextlevel
  \action*{Client knows 1-RTT keys; Handshake Complete}{cl}
  \action*{Handshake Complete}{sr}
  \nextlevel[2]
  \mess{\emph{1-RTT [DATA]}}{cl}{sr}
  \nextlevel[2]
  \mess{\emph{1-RTT [ACK]}}{sr}{cl}
  \nextlevel
  \action*{Handshake Confirmed; Drop Handshake Keys}{cl}
  \nextlevel[2]  
  \mess{\emph{MIGRATION, 1-RTT [DATA]}}{cl}{sr}  
  \nextlevel  
  \action*{Handshake data still outstanding!}{sr}
  
  \nextlevel[2]
\end{msc}
}
\end{center}
\caption{Outstanding handshake data during migration}
\label{fig:outstanding-handshake-migration}
\end{figure}

The issue here is that the server's signal that the handshake
is done is an ACK of its 1-RTT data, but there's no guarantee
that this precedes the client's migration data, because the
client can migrate as soon as it has received an ACK of its
1-RTT data. Even if the server sends its own 1-RTT data in
the same packet as its ACK, there is no requirement that
the client ACK at the same time as it sends its migration
data.

Note that the converse case cannot happen: the client cannot
initiate migration until handshake confirmed and it is required
to drop the handshake keys at that point.

\textbf{QUESTION}: Do current implementations (especially clients)
ACK handshake packets prior to handshake confirmed? If so,
this case ought to be quite rare.


\section{Never Drop the Keys (PR 3121)}
\label{sec:never-drop-keys}

The basic idea here is to (1) never drop the Handshake keys
and (2) continue to send ACKs for packets you have received
(as agreed upon separately in YUL, you MUST ACK each
packet at least once). The result is that the handshake
goes quiescent even if no application data is ever transmitted,
though you will not reach ``handshake confirmed'' in this
case, as shown in Figure~\ref{fig:never-drop-1}.

\begin{figure}[H]
\begin{center}
\resizebox{0.8\columnwidth}{!}{\begin{msc}{}
  \drawframe{none}
  \setlength{\topheaddist}{0cm}
  \setlength{\instdist}{8.5cm}
  \setlength{\instwidth}{1.75cm}
  \declinst{cl}{}{Client $C$}
  \declinst{sr}{}{Server $S$}

  \mess{\emph{Initial [C]}}{cl}{sr}
  \nextlevel[2]
  \mess{\emph{Initial [SH], Handshake[EE...SFIN]}}{sr}{cl}
  \nextlevel
  \action*{Server knows 1-RTT keys}{sr}
  \nextlevel[2]
  \mess{\emph{Handshake [CFIN, ACK]}}{cl}{sr}
  \nextlevel
  \action*{Client knows 1-RTT keys; Handshake Complete}{cl}
  \action*{Handshake Complete}{sr}
  \nextlevel[3]
  \mess{\emph{Handshake [ACK]}}{sr}{cl}
  
  \nextlevel[2]
\end{msc}
}
\end{center}
\caption{Never drop keys (no data sent)}
\label{fig:never-drop-1}
\end{figure}


One obvious question is what happens in the case where you
\emph{are} sending data. In the basic case, we would just
expect ACK piggybacking on the data, as shown in Figure~\ref{fig:never-drop-2}

\begin{figure}[H]
\begin{center}
\resizebox{0.8\columnwidth}{!}{\begin{msc}{}
  \drawframe{none}
  \setlength{\topheaddist}{0cm}
  \setlength{\instdist}{8.5cm}
  \setlength{\instwidth}{1.75cm}
  \declinst{cl}{}{Client $C$}
  \declinst{sr}{}{Server $S$}

  \mess{\emph{Initial [C]}}{cl}{sr}
  \nextlevel[2]
  \mess{\emph{Initial [SH], Handshake[EE...SFIN]}}{sr}{cl}
  \nextlevel
  \action*{Server knows 1-RTT keys}{sr}
  \nextlevel[2]
  \mess{\emph{Handshake [CFIN, ACK]}}{cl}{sr}
  \nextlevel
  \action*{Client knows 1-RTT keys; Handshake Complete}{cl}
  \action*{Handshake Complete}{sr}
  \nextlevel[2]
  \mess{\emph{1-RTT [DATA]}}{cl}{sr}
  \nextlevel[2]
  \mess{\emph{Handshake [ACK], 1-RTT [DATA, ACK]}}{sr}{cl}
  \nextlevel
  \action*{Handshake Confirmed}{cl}
  \nextlevel[3]  
  \mess{\emph{1-RTT [ACK]}}{cl}{sr}
  \nextlevel    
  \action*{Handshake Confirmed}{sr}
  \nextlevel[2]
  
  \nextlevel[2]
\end{msc}
}
\end{center}
\caption{Never drop keys (data sent)}
\label{fig:never-drop-2}
\end{figure}

You can also safely implement implicit ACK. At the point
where you have handshake confirmed, you can stop sending
any of your Handshake DATA frames (though you still
need to ACK any of the peer's Handshake frames, or you
will get deadlocks if the peer doesn't implement implicit
ACKs.)

Kazuho had raised two concerns about this design:

\begin{enumerate}
\item Because handshake confirmed is not guaranteed, key update
  is confusing.
\item It is possible to have outstanding handshake data at the
  point where you want to do migration.
\end{enumerate}

I don't really understand point (1). As noted in
\url{https://github.com/quicwg/base-drafts/issues/3212}, the handshake
confirmed test for key update is redundant, and so we can just re-code
this as ``don't initiate key update for key $n+1$ until you are sure
the other side has key $n$''. And this just lets us have the usual
rule without a special case for the handshake $\rightarrow$ 1-RTT
transition.

Point (2) is the same issue as is raised in
Section~\ref{sec:outst-handsh-data} above. It is possible that this
case is less likely with the ``never drop keys'' approach because
the client is required to send ACKs for handshake data. Thus, if
the client piggybacks ACKs for the server's handshake packets
on its own CFIN (which is natural), the server will never be
able to process 1-RTT packets from the client when handshake
data is outstanding, in which case the problem goes away. Note
that this is separate from the packet loss edge cases, because
the client can guarantee this condition.

An alternative approach is to have the server implement
implicit ACK where it discards handshake data once it receives
any valid 1-RTT data from the client. Note that it still needs to
be able to receive handshake data and send ACKs for it, but
the client can be required to send that down the initial
path and/or implicitly ACK on migration success (which inevitably
includes verification of a 1-RTT ACK).



\end{document}

